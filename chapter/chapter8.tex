% !TeX root = ../main.tex

\chapter{Summary}

In above research, the silicon nitride ring cavities are fully studied as a source of frequency entangled photons. 

To summarize up, in \autoref{chap:3}, dispersion compensation method for silicon nitride ring resonators was established thus gave the dimension requirement on waveguide cross section. 
In our conclusion, a 1.5 \um wide and 0.8 \um thick waveguide is suitable for zero dispersion at 1550 nm.
Following this dimension, subtractive fabrication process were performed especially with films deposited using different CVD methods in \autoref{chap:4}. 

In \autoref{chap:5}, the material properties of films used above is first studied using ellipsometry and Fourier-transform infrared spectroscopy (FTIR). Then all the fabricated device were estimated and compared in the term of \textit{Q}-factors, FSR and mode dispersion. 
Transmission spectra of these devices show absorption around 1530 nm, which agree with the result of absorbance from FTIR.
The highest \textit{Q}-factor is up to \num{5d4} observed in samples using liquid source CVD. In the dispersion evaluation, our samples show highly agreement between the designed mode dispersion and measured values. In our future fabrication, such design is promising to realize broadband frequency entangled photon pair generation.

Furthermore, the fabless samples were introduced in \autoref{chap:6} and featured high \textit{Q}-factor up to \num{d6}. Thus, we focused on the research of photon pair generation using Group 2 Device 1 in \autoref{chap:7}. The single photon flux is around \num{d6} cps/mW and coincidence count is \num{d4} cps/mW based on our measurement setup. By evaluating joint spectral intensity, 46 mode pairs show frequency correlation using 24.5 mW pump power.

\newpage

In general, silicon nitride ring cavities behave outstanding performance as the source of frequency entangled photon pairs.
%By future wavelength division 
Such devices also prove useful in expanding our understanding of how dispersion effect the pair generated in the term of broadband.

\bigskip

Here, we hope to give some future perspective on this topic. 

First is on the device fabrication. Relative research like Kerr frequency combs offered some important insights of fabrication skills concerning dry etching and film tensile control. 

Second, theoretically, the dispersion of ring cavities can be optimized into fully zero and flat from visible and infrared range, using computational iteration. But the photonic crystal could also be a powerful approach since the photonic band structure is able to confine the dispersion more efficiently referring to the research on slow-light generation. 

Finally, in the field of optical quantum information processing, adding active devices such as electro-optic modulators can lead to manipulable frequency entanglement, which paves the way for future frequency-encoded optical quantum information technology.