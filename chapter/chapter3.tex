% !TeX root = ../main.tex

\chapter{Phase match condition for spontaneous four wave mixing in a ring cavity }\label{chap:pmc-sfwm}

According to the previous chapter, in a typical nonlinear optical waveguide or silica fibers, despite the stimulated Raman and Brillouin scattering, the frequency conversion processes involve not only the self-phase modulation of pump light and cross-phase modulation of signal and idler light, but also the phase mismatch in four wave mixing propagation factor. In this case, it is necessary to study the coupled nonlinear equations involving signal, idler and pump intensity \cite{AGRAWAL2013397}. 

Whereas in ring resonators, whose mode linewidth (pm) is much narrower than SPM frequency broadening, 
the frequency broadening in single mode is approximately negligible. 
Thus the phase mismatch among cavity modes becomes the critical factor of the band of four wave mixing.

This chapter first describes the major origin of phase mismatch, chromatic dispersion, and goes on to the design philosophy used in device fabrication. Besides, several topics concerning the band of phase matching are also included.

\section{Chromatic dispersion}

In a typical FWM process, both energy conservation and momentum conservation are required 
\begin{align}
  \beta_i + \beta_s & = 2 \beta_p \label{eq:beta-cons} \\
  \omega_i + \omega_s & = 2 \omega_p \label{eq:omega-cons}
\end{align}
where the subscripts $s~i~p$ stand for signal, idler and pump light.

Meanwhile, the resonance condition \autoref{eq:res-con} gives $\beta = m \frac{2 \pi}{L}$. Thus, \autoref{eq:beta-cons} is equivalent to 
\begin{equation}\label{eq:mu-cons}
  m_i + m_s = 2 m_p 
\end{equation} 
  
We can see that \tmtextit{the momentum conservation agrees with mode number
conservation.} That is to say, by choosing the equidistant mode relative the pump mode, the momentum conservation can be naturally satisfied, which is the most important difference from non-resonant devices.

Therefore, we can estimate the phase mismatch only in the frequency domain. Expand the frequency into Taylor seires at $\omega_0$ to the propagation constant $\beta$
\begin{eqnarray}
  \omega_{\mu} & = & \omega_0 
  + \sum_{j=1}\dv[j]{\omega}{\beta} (\beta_{\mu}-\beta_0) \\
  & = & \omega_0 
  + \sum_{j=1}\dv[j]{\omega}{\beta} \qty(\frac{2 \pi}{L})^j \mu^j \nonumber \\
  & = & \omega_0 + D_1 \mu + D_2 \mu^2 + D_3 \mu^3 + \cdots \nonumber
\end{eqnarray}
where $D_j \equiv  (\frac{2 \pi}{L})^j\dv*[j]{\omega}{\beta}$ are \textit{j}-order mode number dispersion parameter, whose dimension are all $\mathrm{T}^{-1}$ and $\mu \in \mathbb{Z}$ is the relative mode number. 

It is easy to know that $D_1 / 2 \pi = v_g / L$ is the free spectral range in the frequency and indicates that the dispersion property is related with the difference of resonant frequencies.   

Next, we introduce the integrated dispersion $\dint$ \cite{Brasch2014a} to analyse the phase mismatch
\begin{align}\label{eq:def-dint}
    D_\mathrm{int}(\mu) &\equiv \omega_{\mu} - (\omega_0 + D_1 \mu)  \\
    &= D_2 \mu^2 + D_3 \mu^3 + \cdots \nonumber
\end{align}

In particular, $\dint$ includes the residual dispersion higher than second order. Approximately, if $D_3 \mu \ll D_2$, the second-order dispersion will dominate the phase mismatch both at signal and idler mode.

Indeed, the mode number dispersion parameter is linked with the dispersion coefficients in frequency and wavelength domain, giving such a chain rule
\begin{equation}\label{eq:disp-chain}
    D_2 = - \frac{L}{2\pi} {D_1^3}{\beta_2} = \frac{L}{2\pi}  \frac{\lambda^2}{2\pi c} {D_1^3} D_{\lambda}
\end{equation}
where $\beta_2=\dv*[2]{\beta}{\omega}$ is group velocity dispersion (GVD) and $D_{\lambda}=(\lambda/c)\dv*[2]{n}{\lambda}$ is the dispersion parameter.

From the above derivation, a rough presupposition to increase the efficient phase matched band is achieving zero and flat dispersion around pump wavelengths.

\section{Dispersion compensation}
Mentioned previously in \autoref{sec:guide}, the dispersion behaviour in integrated devices is not only the intrinsic material property, but also depends on the waveguide dimension. In other words, the phase mismatch occurs as a result of material dispersion and waveguide dispersion, $D= D_M + D_W$
In the case of silicon nitride, 

This indicates that the dispersion 

Usually, the Sellmeier equation is used to fit the refractive index for a particular transparent medium. \citeauthor{Luke2015a} reported the measured refractive index of stoichiometric \ce{Si3N4}
\begin{equation}\label{eq:si3n4-selleimeier}
    n_{\ce{Si3N4}}^2 = 1 + \frac{3.0249 \lambda^2}{\lambda^2-135.3406^2} + \frac{40314 \lambda^2}{\lambda^2 - 1239842^2}
\end{equation}
It is plotted in , along with the dispersion parameter, defined as $D_{\lambda}=(\lambda/c)\dv*[2]{n}{\lambda}$

On the other hand, the numerical simulation is adopted to evaluate the dispersion parameter, then according to the second-order dispersion chain rule \autoref{eq:disp-chain}, 



structure dispersion

\begin{figure}
    \centering
    \includesvg{1550}
    \caption{Caption}
    \label{fig:my_label}
\end{figure}

\section{Dispersion engineering using slot structure}

\section{Effects of mode crossing}


Using the commercial Comsol Multiphysics package for FEM simulations, we implemented a 2D simulation which takes into account the cylindrical symmetry of the system in the third dimension. Material dispersion is taken into account via an iterative approach which takes the values of the refractive index from measured values for our SiN films. The values for dispersion parameters are obtained from fitting appropriate polynomials to the absolute frequencies of the modes around our pump wavelength of 192.2 THz. The relative magnitude of the values for the free spectral range in the simulation is used to identify the mode