\chapter{Introduction}
% Quantum mechanics has not only boosted up the modern science and technology in different disciplines, but also established the cornerstone for future quantum information processing technology. Furthermore, differing from the past applications, which only involves the quantization nature, the quantum information processing, including but not limited to  quantum computation, quantum communication and quantum metrology, exploits the deep-level physics of quantum mechanics, such as quantum superposition and quantum entanglement.

% Meanwhile, 
The flying qubit---photon---featuring the advantages of long coherent time and multiple degrees of freedom (DoF), is a promising candidate in quantum computation and quantum communication. However, in the term of DoF, much of research up to now focuses on the photon polarization and path entanglement realization, very little attention has been paid to the role of frequency entanglement, which is continuous and infinite in hilbert space. 

Furthermore, the previous research shows that frequency entangled photon pairs can be exploited not only in wavelength division multiplexing quantum key distribution \cite{Wengerowsky2018} but also transferring quantum information in future quantum networks \cite{Tchebotareva2019}. Besides, in recent applications of quantum metrology, quantum optical coherent tomography (QOCT), the broadband frequency entanglement \cite{Okano2015} is also required. 

\section{Background}

% The state-of-art method to generate frequency entangled photon pair is to use bulky crystals.
Compared with crystal experiments, a chip-scale photon source has the advantages of scalability and robustness. A conventional material candidate is silicon on insulator (SOI), since it is CMOS-compatible and supplied by a lot of wafer manufacturers. However, suffering from two photon absorption and stimulated Raman scattering, silicon is no able to generate broadband frequency conversion. Since that, silicon nitride, which is transparent from visible to near infrared range, is preferred to perform broadband frequency entanglement. A recent record is single pair from visible to telecom band \cite{Lu2016}.

\section{Objective}
In the platform of silicon nitride, we utilize the third-order optical nonlinearity and confine the light in the sub-micron scale---optical waveguides. To enhance the nonlinear interaction, we define the ring resonator and couple the light inside along with a bus waveguide. The broadband property is ensured by carefully optimizing the device geometry to achieve a broadband phase matching condition. All the photon pair generation events are detected by single photon detectors and verified by the coincident counting.

In this dissertation, the terms 'broadband photon pair' is used to describe that pairs are generated in different frequency pairs simultaneously and but entangled only in the single mode pair. 

\section{Outline}
The following chapters are sequentially divided in different topics.

