\chapter{Introduction}
Quantum mechanics has not only boosted up the modern science and technology in different disciplines, but also established the cornerstone for future quantum information processing technology. Furthermore, differing from the past applications, which only involves the quantization nature, the quantum information processing, including but not limited to  quantum computation, quantum communication and quantum metrology, exploits the deep-level physics of quantum mechanics, such as quantum superposition and quantum entanglement.

Meanwhile, the flying qubit---photon---featuring the advantages of long coherent time and multiple degrees of freedom (DoF), is a promising candidate in quantum computation and quantum communication. However, in the term of DoF, much of research up to now focuses on the photon polarization and path entanglement, very little attention has been paid to the role of frequency entanglement, which is continuous and infinite in hilbert space. 

Furthermore, the previous research shows that frequency entanglement can be used not only in multiplexing quantum key distribution \cite{Wengerowsky2018} but also transferring quantum information in future quantum networks \cite{Tchebotareva2019}. On the other hand, in recent applications of quantum metrology, quantum optical coherent tomography (QOCT), the broadband frequency entanglement \cite{Okano2015} is also required. 

In conclusion, the specific objective of this study was to realize broadband frequency entangled photon generation. 


\section{Background}

\subsection{Chip-scale quantum information processing}
\subsection{Entangled photon sources for quantum communication and sensing}

\section{Current problems}